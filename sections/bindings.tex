\section{Syntax Bindings}

\subsection{General Overview}
\begin{multicols*}{2}
	Since every expression in Anterogradia must either be a string literal or a function call, \textbf{syntax bindings} were
	introduced in order to solve the readability and practicality issues.
	Syntax bindings are nothing more than fancy syntactical entities that are directly translated into standard library function
	calls by the parser.
	Take a look at the following piece of code as an example:
	\begin{verbatim}
		progn {
		    fun sayHi {
		        "Hello, World!"
		    }
		    eval sayHi
		}
	\end{verbatim}
	This code is simple enough to be able to almost immediately notice the two obvious primitive expressions:
	\begin{enumerate}
		\item The \textbf{variadic function} call of \textit{progn}
		\item The string literal \verb|"Hello, World!"|
	\end{enumerate}
	However, here, \verb|fun| and \verb|eval| don't match the established syntax for any ANTG primitive.
	Thus, you'd be correct to conclude that they are in fact syntax bindings.
	With the help of the \textit{astd} (AST Dump) function we can take a peek behind what's going in the program above:
	\begin{verbatim}
		astd(expr = progn { ...
	\end{verbatim}
	Yields:
	\begin{verbatim}
		progn {
		    _fun(
		        expr = progn {
		            "Hello, World!"
		        },
		        id = "sayHi"
		    ),
		    _eval(id = "sayHi")
		}
	\end{verbatim}
	Here you can see that the \verb|fun| entity has been translated to a \verb|_fun| call with a \verb|progn| call as its
	return value.
	The function identifier is now also provided as a discrete parameter.
	As for the \verb|eval| entity, it was also changed a bit by turning into an \verb|_eval| call with the function id
	as its parameter.
	The difference is not particularly striking in this example, but it's enough to establish what this technique is all about.
	Syntax bindings really start to shine when your code samples slightly grow in complexity, as illustrated by the following example:
	\vfill\columnbreak\noindent
	Here, the original code $\dots$
	\begin{verbatim}
		progn {
		  fun sayHi <to> {
		    sequence { "Hello, " &`to "!" }
		  }
		  eval sayHi(to = "World")
		}
	\end{verbatim}
	$\dots$ expands to $\dots$
	\begin{verbatim}
		progn {
		  _fun(
		    expr = progn {
		      __require_prop(
		        err = "Required prop not present",
		        id = "to"
		      ),
		      progn {
		        sequence {
		          "Hello, ",
		          get(key = "to"),
		          "!"
		        }
		      }
		    },
		    id = "sayHi"
		  ),
		  progn {
		    set(value = "World", key = "to"),
		    _eval(id = "sayHi")
		  }
		}
	\end{verbatim}
	To briefly summarize what the parser has done, the most noticeable change is the transformation of would-be
	discrete function parameters to variable declarations prior to evaluating the stored function.
	The parser also generates safeguards at the beginning of the function body to ensure that all required
	variables are in fact present; this is achieved via the \verb|__require_prop| function, which checks
	for the existence of a variable \textit{id} and causes the runtime to throw an \verb|AntgRuntimeException|
	with the error message \textit{err} whenever it fails to locate said variable.
	It is also worth mentioning that since the parser has no notion of functions and variables (after all, this functionality
	is implemented in the standard library) there is no way for it to check the validity of the parameters passed to the
	\verb|eval| entity, and thus it will transform any discrete parameters into variable declarations, regardless of whether
	they're actually required by the callee.
\end{multicols*}
\newpage

\subsection{Standard Library Bindings}

\begin{multicols*}{2}
	\docbind{Magnitude operator}{len(expr = "Lorem ipsum")}{|"Lorem ipsum"|}
	\docbind{Conditional expression}{\_if(cond = .., then = "Lorem", else = "Ipsum")}{if (...) \{ "Lorem" \} else \{ "Ipsum" \}}
	\docbind{Function definition without parameters}{\_fun(id = "foo", expr = "Hello, World")}{fun foo \{ "Hello" \}}
	\docbinds{Function definition with required property checks}{fun foo <a, b, ..,> \{ "Hello" \}}
	\docbind{Function call without properties}{\_eval(id = "foo")}{eval foo}
	\docbinds{Function call with required property assignments}{eval foo (a = .., b = ..,)}
	\docbind{Lexical atom to string conversion}{"123"}{`123}\\
	\bindequiv{"foo"}{`foo}\\
	\bindequiv{"("}{`(}
	\docbind{Variable assignment}{set(key = "i", value = "Bar")}{`i := "Bar"}
	\docbind{Variable retrieval}{get(key = "i")}{\&`i}
	\docbind{Equality check}{equal(left = 10, right = 20)}{10 = 20}
	\docbind{Left greater check}{lgt(left = 10, right = 20)}{10 > 20}
	\docbind{Right greater check}{rgt(left = 10, right = 20)}{10 < 20}
\end{multicols*}