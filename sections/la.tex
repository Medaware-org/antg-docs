\section{Linear Algebra Library}
\vspace{1.5mm}
The following documentation assumes this library was imported as \textit{la}:
\begin{verbatim}
	@library "org.medaware.anterogradia.libs.LinearAlgebra" as la
\end{verbatim}
\begin{multicols*}{2}
	\docentry{v2.0.0}{validate}
	Checks whether \textit{str} is a valid vector object.
	Returns \verb|"true"| or \verb|"false"| respectively.
	\begin{verbatim}
		la.validate(str = "1|2")
		    => true

		la.validate(str = "Lorem ipsum")
		    => false
	\end{verbatim}
	\docentry{v2.0.0}{v}
	Creates a vector object from the given dimensions.
	This function has variadic parameters.
	\begin{verbatim}
		la.v { 1 0 2 }
		    => 1.0|0.0|2.0

		la.v { 2 4 }
		    => 2.0|4.0
	\end{verbatim}
	\docentry{v2.0.0}{sum}
	A variadic function that returns the sum of the given vectors.
	All vectors are required to be of the same dimensions.
	\begin{verbatim}
		la.sum { la.v { 1 1 } la.v { 2 3 } }
		    => 3.0|4.0
	\end{verbatim}
	\docentry{v2.0.0}{sub}
	A variadic function that returns the result of subtracting the given vectors left to right.
	All vectors are required to be of the same dimensions.
	\begin{verbatim}
		la.sub { la.v { 1 1 } la.v { 2 3 } }
		    => -1.0|-2.0
	\end{verbatim}
	\docentry{v2.0.0}{mul}
	Multiplies the vector \textit{v} by a scalar factor \textit{fac}.
	\begin{verbatim}
		la.mul(v = la.v { 1 0 }, fac = 5)
		    => 5.0|0.0
	\end{verbatim}
	\docentry{v2.0.0}{div}
	Divides the vector \textit{v} by a scalar divisor \textit{div}.
	\begin{verbatim}
		la.div(v = la.v {10 5}, div = 2)
		    => 5.0|2.5
	\end{verbatim}
	\columnbreak
	\docentry{v2.0.0}{normalize}
	Normalizes the vector \textit{v} by dividing each component by the vector magnitude.
	\begin{verbatim}
		la.normalize(v = la.v { 4 3 })
		    => 0.8|0.6
	\end{verbatim}
	\docentry{v2.0.0}{len}
	Computes the magnitude (Euclidean norm) of the vector \textit{v}.
	\begin{verbatim}
		la.len(v = la.v { 4 3 })
		    => 5.0
	\end{verbatim}
	\docentry{v2.0.0}{x}
	Retrieves the X dimension (the 0\textsuperscript{th} dimension) from the vector \textit{v}
	\begin{verbatim}
		la.x(v = la.v { 1 2 3 })
		    => 1.0
	\end{verbatim}
	\docentry{v2.0.0}{y}
	Retrieves the Y dimension (the 1\textsuperscript{st} dimension) from the vector \textit{v}
	\begin{verbatim}
		la.y(v = la.v { 1 2 3 })
		    => 2.0
	\end{verbatim}
	\docentry{v2.0.0}{z}
	Retrieves the Z dimension (the 2\textsuperscript{nd} dimension) from the vector \textit{v}
	\begin{verbatim}
		la.z(v = la.v { 1 2 3 })
		    => 3.0
	\end{verbatim}
	\docentry{v2.0.0}{n}
	Retrieves the \textit{n}\textsuperscript{th} dimension from the vector \textit{v}
	\begin{verbatim}
		la.n(v = la.v { 1 2 3 4 }, n = 3)
		    => 4.0
	\end{verbatim}
	\docentry{v2.0.0}{dot}
	Computes the dot product (``scalar product'') between the vectors \textit{a} and \textit{b}
	\begin{verbatim}
		la.dot(a = la.v { 1 2 }, b = la.v { 4 3 })
		    => 10.0
	\end{verbatim}
\end{multicols*}